%%%%%%%%%%%%%%%%%%%%%%%%%%%%%%%%%%%%%%%%%
% Long Professional Curriculum Vitae
% LaTeX Template
% Version 1.1 (9/12/12)
%
% This template has been downloaded from:
% http://www.latextemplates.com
%
% Original author:
% Rensselaer Polytechnic Institute (http://www.rpi.edu/dept/arc/training/latex/resumes/)
%
% Important note:
% This template requires the res.cls file to be in the same directory as the
% .tex file. The res.cls file provides the resume style used for structuring the
% document.
%
%%%%%%%%%%%%%%%%%%%%%%%%%%%%%%%%%%%%%%%%%

%---------------------
%	PACKAGES AND OTHER DOCUMENT CONFIGURATIONS
%---------------------

\documentclass[11pt]{res} % Use the res.cls style, the font size can be changed to 11pt or 12pt here
\usepackage[russian]{babel}
\usepackage{hyperref}
\usepackage{biblatex}
\usepackage{helvet} % Default font is the helvetica postscript font
%\usepackage{newcent} % To change the default font to the new century schoolbook postscript font uncomment this line and comment the one above

\newsectionwidth{0pt} % Stops section indenting

\begin{document}

%--------------------------------------------------------
%	YOUR NAME AND ADDRESS(ES) SECTION
%--------------------------------------------------------

\name{\large{Lorenzo Campoli}}

\address{{\bf Personal details} \\ 
Date of birth: 05/04/1986 \\
Nationality: Italian \\
Email: campoli.lorenzo@gmail.com \\
Phone: +79818367326 \\
Skype: lorenzowz \\ 
Github: \url{https://github.com/lkampoli} \\
ORCID: \url{https://orcid.org/0000-0002-0510-9422} \\
ResearchGate: \url{https://www.researchgate.net/profile/Lorenzo\_Campoli} \\
LinkedIn: \url{https://www.linkedin.com/in/lorenzo-campoli-325299191}}

\address{} 

%----------------------------------------------------------------------------------------

\begin{resume}

%----------------------------------------------------------------------------------------
%	PROFESSIONAL EXPERIENCE SECTION
%----------------------------------------------------------------------------------------

{\sl Assistant Professor} \hfill  Jun. 2020 - present \\
Saint Petersburg State University, Faculty of Mathematics and Mechanics (\url{http://gam.spbu.ru}) \\
\vspace{-0.4cm}
\begin{itemize} \itemsep -2pt
\item Research activity (computational fluid dynamics and machine learning code development for non-equilibrium flows in state-to-state and multi-temperature formulations)
\item Teaching activity (see section below)
\end{itemize}

\section{{Previous work experience}}

{\sl Assistant researcher} \hfill  Nov. 2018 -- Apr. 2020 \\
Saint Petersburg State University, Faculty of Mathematics and Computer Science \\ (\url{https://chebyshev.spbu.ru}) \\
\vspace{-0.4cm}
\begin{itemize} \itemsep -2pt
\item Numerical modelling and investigation of viscous finger phenomenon and enhanced recovery methods for flows in porous media.
\end{itemize}
\vspace{-0.4cm}
Supervisors: Prof.  S. Tikhomirov

{\sl PostDoc researcher} \hfill  Nov. 2017 -- Nov. 2020 \\
Saint Petersburg State University, Faculty of Mathematics and Mechanics (\url{http://gam.spbu.ru}) \\
\vspace{-0.4cm}
\begin{itemize} \itemsep -2pt
\item Code developement for high speed non-equilibrium reacting flows with state-to-state
approach.
\end{itemize}
\vspace{-0.4cm}
Supervisors: Prof.  E. Kustova

{\sl Visiting researcher} \hfill  Jun. -- Jul. 2015 \\
NASA Ames, Wescoat Rd, Mountain View, CA 94043, USA (\url{www.nasa.gov/centers/ames/home}) \\
\vspace{-0.4cm}
\begin{itemize} \itemsep -2pt
\item Implement a boundary version of the shock-fitting technique onto the high-order finite differences NASA code ADPDIS3D
\end{itemize}
\vspace{-0.4cm}
Supervisors: Dott. H. C. Yee, Dott. D. Kotov and Prof. B. Sjogreen

{\sl Visiting researcher} \hfill Mar. -- Jun. 2015 \\
INRIA, 200 Avenue de la Vieille Tour, 33405 Talence, France (\url{www.inria.fr/centre/bordeaux}) \\ 
\vspace{-0.4cm}
\begin{itemize} \itemsep -2pt
\item Coupling of a shock-fitting algorithm with the Residual Distribution (RD) INRIA code RD-RK2 for the study of unsteady two-dimensional flows on moving and deforming unstructured grids with an Arbitrary Lagrangian-Eulerian (ALE) formulation
\end{itemize}
\vspace{-0.4cm}
Supervisors: Dott. M. Ricchiuto and Prof. R. Paciorri

\vspace{0.2cm}

{\sl Research assistant} \hfill May -- Dec. 2013 \\
AVIO, Corso G. Garibaldi 22, 00034, Colleferro, Italy (\url{www.aviogroup.com}) \\
\vspace{-0.4cm}
\begin{itemize} \itemsep -2pt
\item Pressure oscillations numerical simulations of P80 Vega Solid Rocket Motor (SRM)
\item Aerothermodynamic numerical simulations on Vega launcher fairing
\end{itemize}
\vspace{-0.4cm}
Supervisor: Dott. F. Paglia

{\sl Technical consultant} \hfill  Feb. -- Apr. 2013 \\
NHAZCA, Via Cori snc, 00177, Rome, Italy (\url{www.nhazca.it}) \\
\vspace{-0.4cm}
\begin{itemize} \itemsep -2pt
\item Algorithms development and implementation for terrestrial and satellite interferometric radar applications (SAR, TinSAR, DinSAR)
\end{itemize}

{\sl Stagist} \hfill  Jun. -- Jul. 2004 \\
INFN, Via E. Fermi, 00044, Frascati, Italy (\url{www.infn.it}) \\
\vspace{-0.4cm}
\begin{itemize} \itemsep -2pt
\item Homogeneity measurements of a MultiWire Proportional Chamber (MWPC)
\end{itemize}
\vspace{-0.4cm}
Supervisor: Dott. M. Anelli

%----------------------------------------------------------------------------------------
%	TEACHING EXPERIENCE SECTION
%----------------------------------------------------------------------------------------

\section{{Teaching experience}} 
Saint Petersburg State University, Faculty of Mathematics and Mechanics (\url{https://gam.spbu.ru})\\
\vspace{-0.4cm}

\begin{itemize} \itemsep -15pt
\item Hypersonics \hfill Sep. 2020 -- Sep. 2022 \\
\item Machine Learning for Fluid Mechanics \hfill Mar. 2021 -- Sep. 2022 \\
\item Modern Scientific Visualization \hfill Mar. 2021 -- Jun. 2021 \\
\item Scientific Paper Writing \hfill Mar. 2021 -- Jun. 2021 \\
\item Concepts of Modern Natural Science \hfill Mar. 2021 -- Jun. 2021
\end{itemize}

%----------------------------------------------------------------------------------------
%	EDUCATION SECTION
%----------------------------------------------------------------------------------------

\section{{Degrees}}
{\sl Ph.D.} 
in Theoretical and Applied Mechanics \hfill Nov. 2013 -- Nov. 2016 \\ 
Sapienza University of Rome (Dept. of Mechanics and Aeronautics)\\
Thesis:  A shock-fitting technique for two-dimensional unsteady flows on unstructured grids \\
%Date of the defense of the Ph.D.: 08/02/2017 \\
%Host institution: Department of Theoreticals and Applied Mechanics (Sapienza)\newline
%Description: A novel, unstructured, shock-fitting algorithm capable of simulating steady flows has being further developed to make it capable of dealing with unsteady flows with different gasdynamics solvers. It is shown that this technique allows to compute numerical solutions that converge, both pointwise and in a global sense, with an observed order of accuracy that is very close to the design order of the spatial discretization scheme and with very small discretization errors. Topological variations occurring during the unsteady flow field evolution have been treated indentifying primitive operations to be performed. \\
Date of the defense of the Ph.D.: 08/02/2017 \\
Supervisor: Prof. R. Paciorri  \\
Advisor: Prof. C. M. Casciola

{\sl Second level Master}  
in Space Transportation Systems  \hfill Feb. 2012 -- Dec. 2012 \\ 
Sapienza University of Rome\\
https://web.uniroma1.it/mastersts \\
Thesis: Solid rocket motor's internal fluid dynamics: fluid-structure interaction \\ 
%Description: In the present work, solid rocket motors' (SRM) internal fluid dynamics was numerically addressed, with a particular emphasis on the interaction between frontal thermal protection (PTF) and fluid flow. Specifically, under the fluid dynamic action of the internal field of velocity, the PTF's surface is supposed to deflect. The final aim of the investigation was to estimate the extent of this deformation relatively to different values of the initial velocity profile. Numerical results, were compared with experimental data provided by Von Karman Institute (VKI) cold flow test facility. \\
Date of the defense of the Master: 20/01/2013 \\
Supervisor: Dott. F. Paglia \\ 
Advisor: Prof. F. Nasuti 
 
{\sl Master} in Space Engineering  \hfill Sept. 2008 -- Feb. 2012 \\ 
Sapienza University of Rome\\
Thesis: Modeling and simulation of base flows in subsonic regime \\
Date of the defense of the Master: 20/02/2012 \\
Supervisor: Prof. R. Paciorri 

{\sl Bachelor} in Aerospace Engineering \hfill Sept. 2005 -- Dec. 2008 \\ 
Sapienza University of Rome\\
Thesis: Ablative materials: types and characterization \\
Date of the defense of the Bachelor: 20/12/2008 \\
Supervisor: Prof. G. Rinaldi

%----------------------------------------------------------------------------------------
%	COMPUTER SKILLS SECTION
%----------------------------------------------------------------------------------------

\section{{Technical skills}}
\textbf{OS:} Linux \\
\textbf{Programming languages:} Fortran, C/C++, Matlab, Mathematica \\
\textbf{Parallel programming models:} MPI, OpenMP, Coarray, CUDA, OpenACC \\
\textbf{Profiling, debugging and performance analysis tools:} gprof, perf, gdb, Valgrind, Vtune, Advisor, Paraver, Scalasca, Extrae, LIKWID, PAPI \\
\textbf{Version-control software:} svn, bzr, git \\
\textbf{Scripting languages:} linux shell, Perl, Python, Jupyter, Colab \\
\textbf{Machine learning and data mining:} TensorFlow, Keras, PyTorch, Pandas, Scikit-learn, Numpy \\
\textbf{Document processors:} Latex, Lyx, Office \\
\textbf{Fluid dynamic solvers:} STAR CCM+, CFD++, Comsol, Dumux, Fluent, OpenFOAM, SU2, COOLFluid, FronTier++, MRST, Overture and in-house developed codes \\
\textbf{Mesh generators} Triangle, Tetgen, Delaundo, Yams, Gmsh, GRUMMP, mmg3d \\
\textbf{Visualization tools:} Tecplot, Gnuplot, VisIt, ParaView, Matplotlib

\section{{Language skills} (Europass Guidelines)}
Italian (mothertongue), English proficiency (C1), upper intermediate of Russian (B2), intermediate Spanish (A2), basic French (A1).

%----------------------------------------------------------------------------------------
%	PUBLICATIONS SECTION
%----------------------------------------------------------------------------------------

\section{{Research outputs}}
L. Campoli, A. Assonitis, M. Ciallella, R. Paciorrib, A. Bonfiglioli, M. Ricchiuto. UnDiFi-2D: an Unstructured Discontinuity Fitting code for 2D grids. Computer Physics Communications, 2021, https://doi.org/10.1016/j.cpc.2021.108202

Campoli, L., Assonitis, A., Ciallella, M., Paciorri, R., Bonfiglioli, A.,  Ricchiuto, M. (2021). UnDiFi-2D: an Unstructured Discontinuity Fitting code for 2D grids. arXiv preprint arXiv:2105.14269.

Campoli, L., Kustova, E., Maltseva, P. (2021). Assessment of machine learning methods for state-to-state approaches. arXiv preprint arXiv:2104.01042.

Campoli, L., Kunova, O., Kustova, E., Melnik, M. (2020). Models validation and code profiling in state-to-state simulations of shock heated air flows. Acta Astronautica. \\
https://doi.org/10.1016/j.actaastro.2020.06.008

Bakharev, F., Campoli, L., Enin, A., Matveenko, S., Petrova, Y., Tikhomirov, S., \& Yakovlev, A. (2020). Numerical Investigation of Viscous Fingering Phenomenon for Raw Field Data. Transport in Porous Media, 1-22. https://doi.org/10.1007/s11242-020-01400-5

L. Campoli, G. Oblapenko, M. Mekhonoshina, E. Kustova: Numerical investigation of hypersonic
non-equilibrium flow around blunt body by COOLFluiD-Kappa coupling. AIP Conference
Proceedings. 31st International Symposium on Rarefied Gas Dynamics (RGD31), 23rd-27th July
2018, University of Strathclyde, Glasgow, UK.

L. Campoli, G. P. Oblapenko and E. V. Kustova. KAPPA: Kinetic Approach to Physical Processes
in Atmospheres library in C++. Computer Physics Communications, 2018, \\ https://doi.org/10.1016/j.cpc.2018.10.016.

Campoli, L., Quemar, P., Bonfiglioli, A., \& Ricchiuto, M. (2017). Shock-fitting and predictor-corrector explicit ALE Residual Distribution. In Shock Fitting (pp. 113-129). Springer, Cham.

A. Bonfiglioli, R. Paciorri, L. Campoli, Unsteady shock-fitting for unstructured grids. International Journal for Numerical Methods in Fluids, 2015.

\section{{Conferences}} 
Campoli, L., Oblapenko, G. P., \& Kustova, E. V. (2019, August). Overview and perspectives of KAPPA library. In AIP Conference Proceedings (Vol. 2132, No. 1, p. 150005). AIP Publishing LLC. https://doi.org/10.1063/1.5119645

L. Campoli, G. Oblapenko, M. Mekhonoshina, E. Kustova: Numerical investigation of hypersonic non-equilibrium flow around blunt body by COOLFluiD-Kappa coupling. 31st Int. Symposium on Rarefied Gas Dynamics (RGD31), 23rd-27th July 2018, University of Strathclyde,
Glasgow, UK.

O. V. Kunova et al. On the implementation of the software library Kappa and its interface with
COOLFluiD. International scientific conference on mechanics “The Eighth Polyakhovs
Reading”. 30th Jan - 2nd Feb. 2018, Saint Petersburg, Russia.

A. Bonfiglioli, R. Paciorri, L. Campoli, V. De Amicis, M. Onofri, Development of an unsteady
Shock-fitting technique for unstructured grids. 30th International Symposium on Shock Waves
(ISSW30), July 19-24, 2015, Tel-Aviv, Israel.

A. Bonfiglioli, R. Paciorri, L. Campoli, Unstructured shock-fitting calculations of transonic turbo-machinery flows. Proceedings of 11th European Conference on Turbomachinery Fluid dynamics \& Thermodynamics ETC11, March 23-27, 2015, Madrid, Spain.

A. Bonfiglioli, R. Paciorri, L. Campoli, (2014). An unsteady shock-fitting technique for unstructured grids. In Onate, E., Oliver, J., and Huerta, A., editors, 6th. European Congress on Computational Fluid Dynamics (ECFD VI), pages 4864--4872, Barcelona, Spain. ECCOMAS, International Center for Numerical Methods in Engineering.

\vspace{8pt}
M. Onofri, R. Paciorri, L. Campoli, A. Bonfiglioli, An unsteady shock-fitting technique for
unstructured grids. 21st International Shock Interaction Symposium ISIS21, August 3-8, 2014, Riga, Latvia.

R. Paciorri, L. Campoli, Numerical Simulations of Flows Past the IXV Capsule. 5th International ARA Days, May 18-20, 2015, Arcachon, France.

\section{{Research funding and grants}}
Dog\_2019: Разработка инструментов определения определения оптимальных МУН: связь МУН и механизмов воздействия, методы мониторинга \\
% Тихомиров, С. Б., Белов, Ю. С., Баранов, А. Д., Петрова, Ю. П., Ананьевский, А. С., Боровицкий, В. А., Гаваза, К. Г., Енин, А. И., Загороднюк, А. А., Злотников, И. К., Золотов, В. О., Камполи, Л., Ковылина, Е. О., Кумаллагов, Д. З., Крицкий, М. М., Матвеенко, С. Г., Мокеев, А. С., Мостовский, П. А., Платонова, М. В., Пышкин, А. В., Романов, Р. В., Россомахина, М. В., Семенов, А. В., Ставрова, А. К., Фадеева, О. В. & Монаков, Г. В.
%ООО "Газпром нефть": 10 998 180 руб.
13/03/19 → 16/05/19
%Дата гранта: 13/03/19

%2) 
RSF\_RG\_2019 - 1: Моделирование неравновесных течений углекислого газа в современных задачах космической аэродинамики и экологии Земли: 2019 г. этап 1 \\
%Кустова, Е. В., Мехоношина, М. А., Нагнибеда, Е. А., Камполи, Л., Кунова, О. В., Гориховский, В. И., Савельев, А. С., Косарева, А. А., Мельник, М. Ю. & Лукашева, А. А.
%Российский научный фонд
6/05/19 → 31/12/19

%3) 
Dog\_2019: Разработка методов увеличения прогнозной способности трёхмерных цифровых геологических моделей \\
%Тихомиров, С. Б., Баранов, А. Д., Лифшиц, М. А., Платонова, М. В., Ананьевский, А. С., Боровицкий, В. А., Гаваза, К. Г., Иконникова, Е. В., Мостовский, П. А., Апушкинская, Д. Е., Белов, Ю. С., Романов, Р. В., Бахарев, Ф. Л., Енин, А. И., Крицкий, М. М., Матвеенко, С. Г., Россомахина, М. В., Фадеева, О. В., Ковылина, Е. О., Загороднюк, А. А., Давыдов, Ю. А., Гвоздевский, П. Б., Камполи, Л., Лишанский, А. А., Петрова, Ю. П., Растегаев, Н. В., Синчук, С. С., Теплицкая, Я. И., Якубович, Ю. В., Чухров, А. К. & Туник, М. Ю.
%ООО "Газпром нефть"
31/05/19 → 30/09/19

%4)
Dog\_2019: Разработка инструментов определения оптимальных МУН в части модели ПАВ-полимерного и щелочно-полимерного заводнения, оценки Кохв в 5-ти точечной системе разработки, оценка оптимального размера оторочки полимера \\
%Тихомиров, С. Б., Баранов, А. Д., Лифшиц, М. А., Белов, Ю. С., Апушкинская, Д. Е., Романов, Р. В., Бахарев, Ф. Л., Ставрова, А. К., Петрова, Ю. П., Платонова, М. В., Растегаев, Н. В., Боровицкий, В. А., Енин, А. И., Золотов, В. О., Иконникова, Е. В., Ковылина, Е. О., Крицкий, М. М., Кумаллагов, Д. З., Матвеенко, С. Г., Мокеев, А. С., Пышкин, А. В., Россомахина, М. В., Семенов, А. В., Гаваза, К. Г., Загороднюк, А. А., Монаков, Г. В., Мостовский, П. А., Фадеева, О. В., Камполи, Л., Ланцова, М. А., Лишанский, А. А., Затицкий, П. Б., Воронецкий, Е. Ю., Лавренов, А. В., Захаров, А. О., Рядовкин, К. С., Щетка, Е. В. & Туник, М. Ю.
%ООО "Газпром нефть"
22/10/19 → 31/12/19

%5)
Dog\_2019: Разработка методов увеличения прогнозной способности трёхмерных цифровых геологических моделей \\
%Тихомиров, С. Б., Лифшиц, М. А., Азангулов, И. Ф., Алексеев, Я. Ю., Апушкинская, Д. Е., Баранов, А. Д., Бахарев, Ф. Л., Белов, Ю. С., Боровицкий, В. А., Воронецкий, Е. Ю., Гаваза, К. Г., Гвоздевский, П. Б., Давыдов, Ю. А., Енин, А. И., Загороднюк, А. А., Захаров, А. О., Иконникова, Е. В., Калинин, К. М., Камполи, Л., Ковылина, Е. О., Крицкий, М. М., Лавренов, А. В., Ланцова, М. А., Мамаев, Д. А., Матвеенко, С. Г., Мокеев, А. С., Мостовский, П. А., Новиков, С. М., Петрова, Ю. П., Платонова, М. В., Пышкин, А. В., Растегаев, Н. В., Романов, Р. В., Россомахина, М. В., Рядовкин, К. С., Семенишин, Д. А., Семенов, А. В., Сонина, А. К., Тадевосян, А. А., Теплицкая, Я. И., Туник, М. Ю., Фадеева, О. В., Ходунов, П. А. & Якубович, Ю. В.
%ООО "Газпром нефть"
16/12/19 → 19/03/20

%6)
RSF\_RG\_2019 - 2: Моделирование неравновесных течений углекислого газа в современных задачах космической аэродинамики и экологии Земли: 2020 г. этап 2 \\
%Кустова, Е. В., Мехоношина, М. А., Нагнибеда, Е. А., Камполи, Л., Кунова, О. В., Гориховский, В. И., Савельев, А. С., Косарева, А. А., Мельник, М. Ю. & Лукашева, А. А.
%Российский научный фонд
1/01/20 → 31/12/20

%7) 
Dog\_2019: Договор пожертвования №117/19-БП \\
%Баранов, А. Д., Азангулов, И. Ф., Аксенова, Д. Д., Алексеев, И. С., Алексеев, О. В., Алексеев, Я. Ю., Апушкинская, Д. Е., Бахарев, Ф. Л., Бессонов, Р. В., Боровицкий, В. А., Брагилевский, В. Н., Вепрев, Г. А., Вершик, А. М., Воронецкий, Е. Ю., Гаваза, К. Г., Гвоздевский, П. Б., Губкин, П. В., Дородный, М. А., Дружинин, А. Э., Дубашинский, М. Б., Енин, А. И., Загороднюк, А. А., Затицкий, П. Б., Захаров, А. О., Зограф, П. Г., Иванова, О. Я., Иконникова, Е. В., Калинин, К. М., Камполи, Л., Ковылина, Е. О., Крицкий, М. М., Лавренов, А. В., Ланцова, М. А., Лаптев, А., Лишанский, А. А., Логунов, А. А., Мамаев, Д. А., Матвеенко, С. Г., Мнёв, Н. Е., Мокеев, А. С., Монаков, Г. В., Мосеева, Т. Д., Мостовский, П. А., Новиков, С. М., Новожилова, Е. А., Павлов, Д. А., Петрова, Ю. П., Платонова, М. В., Пройссер, Р., Пышкин, А. В., Растегаев, Н. В., Романов, Р. В., Россомахина, М. В., Русских, М. С., Рядовкин, К. С., Семенишин, Д. А., Семенов, А. В., Симарова, Е. Н., Синчук, С. С., Смирнов, С. К., Сонина, А. К., Ставрова, А. К., Старков, И. А., Столяров, Д. М., Тадевосян, А. А., Теплицкая, Я. И., Тихомиров, С. Б., Туник, М. Ю., Фадеева, О. В., Целищев, А. С., Щетка, Е. В., Белов, Ю. С., Пеллер, В. В., Иевлев, П. Н., Нордскова, А. В., Ходунов, П. А., Вайнманн, Т. Р. & Агилар Ривера, М. А.
%Фонд поддержки социальных инициатив "Родные города"
1/01/20 → 31/12/20

%8)
RSF\_RG\_2019 - 3: Моделирование неравновесных течений углекислого газа в современных задачах космической аэродинамики и экологии Земли: 2021 г. этап 3 \\
%Кустова, Е. В., Мехоношина, М. А., Нагнибеда, Е. А., Камполи, Л., Кунова, О. В., Гориховский, В. И., Савельев, А. С., Косарева, А. А., Мельник, М. Ю. & Бечина, А. И.
%Российский научный фонд
1/01/21 → 31/12/21

%9)
M1\_2021 - 1: Машинное обучение в задачах неравновесной аэромеханики: 2021 г. этап 1 \\
%Кустова, Е. В., Михайлова, Е. Г., Графеева, Н. Г., Титарев, В. А., Истомин, В. А., Кунова, О. В., Мехоношина, М. А., Карпенко, А. Г., Камполи, Л., Савельев, А. С., Гориховский, В. И., Бечина, А. И., Мельник, М. Ю., Кива, П. С., Бушмакова, М. А., Лагутин, С. М., Шаламов, И. Ю., Ворошилова, Ю. Н., Богатко, В. И., Добров, Ю. В., Алексеев, И. В., Кравченко, Д. С., Баталов, С. А., Вождаева, Ю. С., Максудова, З. М. & Косарева, А. А.
%Санкт-Петербургский государственный университет
24/08/21 → 31/12/21

%----------------------------------------------------------------------------------------
%	HONORS SECTION
%----------------------------------------------------------------------------------------

\section{{Awards and honours}}

Certificate of participation in several PRACE Training Centre's courses. 
\\ https://training.prace-ri.eu/\\

\vspace{-20pt}
Partnership of a European Group of Aeronautics and Space UniversitieS (PEGASUS), 2012. \\
https://www.pegasus-europe.org

%----------------------------------------------------------------------------------------
%	INTERESTS SECTION
%----------------------------------------------------------------------------------------

\section{{About me}} 

I do sport (running, biking and swimming) on daily basis. I love nature, trekking and alpinism and I was in boy-scout for 10 years. I play classical, electrical guitar and drum.
I'm interested in reading and writing (prose and poetry), astronomy and travelling. I enjoy discovering new cultures, languages, traditions. I'm gratified in helping people.

%----------------------------------------------------------------------------------------

\end{resume} 
\end{document}